\section{Performance Evaluation}\label{sec:pros_perf_eval}

\begin{marginfigure}[-1in]
\centerline{\includegraphics[width=\columnwidth]{pros_with_labels}}
\caption[Prosthesis configuration used for experiments]{Prosthesis configuration
used for experiments. An IMU attached on the thigh measures the thigh angle.
Note: the optional unidirectional ankle springs were not installed for
experiments presented in this thesis as the ankle motor alone produces
sufficient torque for the results presented
herein.}\label{fig:prosthesis_actual}
\end{marginfigure}

\Cref{fig:prosthesis_actual} shows the instantiation of the prosthesis design
used for the results presented in this thesis. As large ankle torques were not
required for the experiments conducted in this thesis, the optional
unidirectional ankle springs were not installed. Of crucial importance for the
work presented in this thesis is that the prosthesis faithfully reproduces the
torques commanded by the mid-level control. This is important so that we can
ensure differences between different mid-level control strategies are due to the
differences in those strategies, and not due to an inability of the prosthesis
to follow the desired behaviors.

We test the ability of the manufactured prosthesis to reproduce desired torques
through three experiments: First, we collect data to construct a Bode plot of
the torque tracking behavior to confirm that the bandwidth of the actuators
exceeds the desired values in \cref{tab:pros_requirements}. Second, we
investigate the ability of the prosthesis to follow the desired torques during a
typical gait stride. And third, we test the ability of the prosthesis to track
zero torque as a function of input disturbance frequency.

\subsection{Torque Tracking Bode Plot}
\begin{figure}[htb]
\centerline{\includegraphics[width=\columnwidth]{bode_fixture}}
\caption[Fixture for testing bandwidth of actuators]{Fixture for testing
bandwidth of actuators. The knee was constrained by a clamp on the shank. The
ankle was constrained by a floor under the foot.}\label{fig:bode_fixture}
\end{figure}
To measure the torque tracking bandwidth of the prosthesis' knee and ankle
actuators, we construct bode plots of the transfer function between measured and
desired torque. For this purpose, we constrain the prosthesis in the fixture
shown in \cref{fig:bode_fixture}. In this fixture, the knee joint is
rotationally constrained by a clamp around the prosthesis' shank and the ankle
joint is constrained by a floor under the foot. To construct the bode plot, we
repeatedly command sinusoidal desired torques of varying frequencies $\omega$
and observe the resulting measured torques. We perform nonlinear least squares
regression to identify the amplitude and phase shift parameters of the function
$f(t) = A \sin(\omega t + \phi)$, where $A$ is the amplitude and $\phi$ is the
phase shift, such that the function fits the desired and measured torques.
Finally, we calculate the Gain, $G(\omega) = 20 \log_{10}
(\nicefrac{A_\tn{meas}}{A_\tn{des}})$ and the phase shift, $\phi(\omega) =
\phi_\tn{meas} - \phi_\tn{des}$, between the measured and desired signals at
each frequency. For both joints, we construct Bode plots at two amplitudes of
desired torque, a common amplitude of $\unit[20]{N \cdot m}$ and at the root
mean square (RMS) torques during stance for that joint.\footnote{$\unit[11.5]{N
\cdot m}$ for the knee, $\unit[55.6]{N \cdot m}$ for the ankle according to data
from \citet{bovi2011multiple} for \unit[80]{kg} person} For the knee joint we
sweep frequencies in the range \unit[1-35]{Hz} in \unit[1]{Hz} increments, and
for the ankle we sweep frequencies in the range \unit[1-20]{Hz}.

\begin{figure}[htb]
    \centering 
    \includegraphics[width=\textwidth]{bode_plot_fixed_labels}
    \caption[Experimentally obtained bode plots of knee and ankle actuator
    torque]{Experimentally obtained bode plots of knee and ankle actuator
    torque. Knee is phase limited at \unit[24]{Hz} while the ankle is gain
    limited at \unit[7]{Hz}.}\label{fig:pros_eval_bode_plots}
\end{figure}
\Cref{fig:pros_eval_bode_plots} shows the resulting knee and ankle bode plots at
the two tested amplitudes. We define the bandwidth of the actuator as the
frequency of the desired torque at which the gain falls below $\unit[-3]{dB}$ or
the phase lag falls below $\unit[-180]{deg}$, whichever comes first. We also
report the lower of the two bandwidth values obtained from the $\unit[20]{N
\cdot m}$ or RMS torques. For the knee joint, we see that the bandwidth is phase
limited at \unit[24]{Hz} by the $\unit[20]{N \cdot m}$ signal. The ankle
bandwidth is gain limited at \unit[7]{Hz} by the RMS stance torque signal. The
measured bandwidth of the knee and ankle actuators is higher than the desired
values based on able-bodied walking data.\footnote{compare
\cref{tab:pros_requirements} desired knee bandwidth from
\citep{sergi2012design}, desired ankle bandwidth from \citet{au2008powered}}

\subsection{Walking Torque Tracking}
Next, we examine the torque tracking performance of the prosthesis during normal
walking. \Cref{fig:torque_tracking_plot} shows the desired and measured torque
during a typical stride at a gait speed of \unitfrac[0.8]{m}{s} while
\cref{tab:torque_tracking_tab} shows the median RMS torque tracking error during
stance and swing over \unit[1]{min} of walking. For both joints, we see that
torque tracking during stance is substantially better than during swing.  This
is the case for two reasons: First, during stance, the load inertia is
significantly increased, as it primarily is comprised of the mass of the user.
Consequently, the SEA's dynamics are slower and easier to control. In contrast,
during swing, the load inertia is primarily the inertia of the prosthesis
itself. This primarily affects the ankle, as the inertia of the foot is
relatively small. Second, during swing the control's primary objective is to
follow a desired swing trajectory, (see \cref{sec:nm_control_prosthesis} for
more details). In this phase, the desired torque is generated by a PD feedback +
feedforward term to track the desired trajectory. Consequently, the desired
torques may not necessarily be feasible if the gains are high, as in the case of
the desired ankle torque during swing.
\begin{figure}[t]
    \centering 
    \includegraphics[width=\textwidth]{torque_tracking}
    \caption[Knee and ankle torque tracking during a typical step]{Knee and
    ankle torque tracking during a typical step. Torque tracking during stance
    is significantly better than during swing due to the significantly increased
    load inertia during stance (see \cref{tab:torque_tracking_tab}). During
    swing, trajectory tracking performance is
    prioritized.}\label{fig:torque_tracking_plot}
\end{figure}
\begin{margintable}[-1in]
    \centering
    \small
    \begin{tabular}{lcc}
        RMS Error (N-m) & Knee & Ankle \\
        \midrule
        Stance & 2.20 & 4.85  \\
        Swing  & 3.82 & 10.15 \\
    \end{tabular}
    \caption{Median root mean squared (RMS) torque tracking error during stance
    and swing}\label{tab:torque_tracking_tab}
\end{margintable}

\subsection{Zero Torque Tracking}
\begin{figure}[b!]
    \centering 
    \includegraphics[width=\textwidth]{zero_torque}
    \caption[Zero torque tracking of the knee and ankle joints]{Zero torque
    tracking of the knee and ankle joints. The prosthesis was fixed to a rigid
    mount and commanded to maintain zero net joint torque while the knee and
    ankle joints were manually oscillated (blue) by hand at an increasingly fast
    rate (purple).  The resulting measured torque is shown in the second row of
    axes in black.}\label{fig:pros_eval_zero_torque}
\end{figure}
Finally, we test the ability of the prosthesis to track the desired torque in
the presence of external disturbances. To test this ability, we removed the
shank clamp and floor from the fixture shown in \cref{fig:bode_fixture} and
commanded zero desired torque to the ankle and knee joints. We then manually
moved each joint in a sinusoidal motion while slowly increasing the frequency.
\Cref{fig:pros_eval_zero_torque} shows the angle of each joint during the
experiment and the approximate frequency of the motion, computed by inverting
the time between peaks of the joint angle. In the second row of plots, we show
the resulting measured torque. At an input disturbance of \unit[2.5]{Hz}, which
is approximately \nicefrac{1}{swing duration}, the knee torque error is
\unit{5.7}{N-m}, which is approximately 20\% of the peak knee swing torque
(compare \cref{fig:torque_tracking_plot}), and the ankle torque error is
\unit{10.9}{N-m}, which approximately equal to the peak ankle swing torque. This
result shows that the knee joint especially should be able to reject
disturbances during swing, which is important due to the more proximal location
of the knee. In contrast, the ankle joint, because of its larger reflected
inertia caused by a 100:1 gear reduction, is less able to reject disturbances at
this frequency.  However, because of its more distal location, the ankle joint
plays a lesser role in responding to disturbances during swing. 
