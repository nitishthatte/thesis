\section{Series Elastic Actuator Control}\label{sec:sea_control}

To control the prosthesis we use Simulink Real-Time (Mathworks, USA), which
samples all sensors (joint encoders, IMU on thigh, force sensors in foot),
computes the desired torques for the knee and ankle joints using a mid-level
control strategy (\cref{sec:back_walking_control}), runs the low-level SEA
control, which calculates the motor velocity needed to achieve these desired
torques, and finally sends the velocity commands to the motor controllers at a
rate of \unit[1]{KHz}. Here, we describe the low-level SEA control.

First, in order to ensure the safety of both the user and the prosthesis, the SEA
controller saturates the desired torque commanded by the mid-level
controller. During normal operation, we allow a maximum of $\unit[\pm 100]{N
\cdot m}$ of torque at the knee and $\unit[\pm 150]{N \cdot m}$ of torque at the
ankle. Furthermore, when either joint passes within \unit[5]{degrees} of its joint
limits, a software joint limit stop is activated that calculates torques that
prevent the actuator from damaging itself. The joint limit torque is inspired by
the limit torque used in the neuromuscular model simulation presented in
\cref{sec:neuro_mech_model} and takes the form
\begin{align}
    \tau_\tn{lim} = k_\tn{lim} \Delta \theta_\tn{lim} 
    (1 + \nicefrac{\dot \theta_\tn{lim}}{\dot \theta_\tn{max}}) 
    (\Delta \theta_\tn{lim}  > 0) 
    \left(\nicefrac{\dot \theta_\tn{lim}}{\dot \theta_\tn{max}} > -1 \right), 
    \label{eq:tau_joint_limit_pros}
\end{align}
where $k_\tn{lim}$ is the joint limit stiffness, $\Delta \theta_\tn{lim}$ is the
penetration of the joint into the limit region, $\dot \theta_\tn{lim}$ is the
penetration velocity, and $\dot \theta_\tn{max}$ defines the retraction
velocity at which the limit torque drops to zero. The knee extension limit
activates below \unit[5]{degrees} of flexion at which point above torque is
added to the desired torque. The knee flexion limit is set to
\unit[90]{degrees} beyond which the limit torque upper bounds the desired knee
torque. For the ankle joint, the flexion and extension limits activate at
$\unit[\pm 20]{degrees}$ beyond which the limit torque upper and lower bound the
desired ankle torque respectively.

Next, to improve the behavior of the prosthesis knee during swing, damping
compensation in proportion to the knee velocity is added to the desired knee
torque to compensate for friction and damping in the knee bearing. Finally,
after saturation and friction compensation, a low pass filter is applied to
ensure the derivatives of the desired torque are smooth and to remove
frequencies that may excite the SEA spring. For both joints we use second-order
Butterworth filters with a cutoff frequency of \unit[25]{Hz} at the knee and
\unit[20]{Hz} at the ankle.

\begin{marginfigure}[-0.0in]
    \centering 
    \includegraphics[height=2in]{sea_diagram_ctrl}
    \caption[Dynamics model used to derive sea control]{Dynamics model used to
    derive sea control. $\theta_m$ is the post-gearbox motor angle, $J_m$ and
    $b_m$ are the reflected motor inertia and damping, $\theta_l$ is the load
    angle, $J_m$ and $b_m$ are the load inertia and damping respectively. $k$ is
    the SEA spring stiffness.  $\tau_\tn{m}$ is the motor torque applied to the
    motor rotor and $\eta$ captures the efficiency of the motor torque
    transmission.}\label{fig:sea_diagram_ctrl}
\end{marginfigure}
Next, to achieve the desired torque $\tau_\tn{des}$ in the SEA spring, we
implement an SEA control similar to that used in
\citet{schepelmann2012development}. The control uses proportional feedback to
calculate the desired velocity needed to achieve the desired load torque:
\begin{align}
    \dot \theta_\tn{des} = k_\tn{p} (\tau_\tn{des} - \tau_\tn{meas})
\end{align}
where $\dot \theta_\tn{des}$ is the desired motor velocity and $\tau_\tn{meas}$
is the measured torque. As shown in \cref{fig:sea_diagram_ctrl}, the torque on
the load can be measured through the deflection of the SEA spring as
$\tau_\tn{meas} = k (\theta_\tn{l} - \theta_\tn{m})$ where $\theta_\tn{l}$ is
the position of the load side of the spring and $\theta_\tn{m}$ is the position
of the motor side. To increase performance, we also add a feed-forward velocity
term to the feedback term. We derive this term by differentiating the desired
torque on the load and solving for the motor velocity
\begin{align}
    \dot \tau_\tn{des} &= k (\dot \theta_\tn{l} - \dot \theta_\tn{m}) \\
    \implies \dot \theta_\tn{m} &= \frac{\dot \tau_\tn{des}}{k} 
        + \dot \theta_\tn{l}.\label{eq:motor_vel_from_tau_des}
\end{align}
Making the total desired velocity,
\begin{align}
    \dot \theta_\tn{des} = k_\tn{p} (\tau_\tn{des} - \tau_\tn{meas}) 
        + \frac{\dot \tau_\tn{des}}{k} + \dot \theta_\tn{l}.
\end{align}

The motor controllers used on the prosthesis (Elmo Motion Control Gold Series)
also accept a feedforward torque command, which can help achieve desired
velocities more quickly. To derive the feedforward motor torque we refer to the
motor dynamics depicted in \cref{fig:sea_diagram_ctrl}. The dynamics of the
motor rotor are
\begin{align}
    \eta \tau_\tn{m} - \tau_\tn{l} - b_\tn{m} \dot \theta_\tn{m} 
        = J_\tn{m} \ddot \theta_\tn{m}
\end{align}
Where $\eta$ is the efficiency of gear set and $\tau_\tn{m}$ is the motor
torque. Solving for the motor torque yields
\begin{align}
    \tau_\tn{m} = \frac{1}{\eta} \left( J_\tn{m} \ddot \theta_\tn{m} 
        + b_\tn{m} \dot \theta_\tn{m} + \tau_\tn{l} \right)
    \label{eq:motor_torque_nosub}
\end{align}
To eliminate the motor velocity and acceleration from the right side of the
equation, we calculate the second derivative of the desired load torque
\begin{align}
    \ddot \tau_\tn{des} &= k (\ddot \theta_\tn{l} - \ddot \theta_\tn{m}) \\
    \implies \ddot \theta_\tn{m} &= \frac{\ddot \tau_\tn{des}}{k} 
        + \ddot \theta_\tn{l},\label{eq:motor_accel_from_tau_des}
\end{align}
and substitute \cref{eq:motor_vel_from_tau_des,eq:motor_accel_from_tau_des} into
\cref{eq:motor_torque_nosub} to arrive at the final feedforward motor torque
\begin{align}
    \tau_\tn{m} = \frac{1}{\eta} \left( J_\tn{m} 
        \left( \frac{\ddot \tau_\tn{des}}{k} + \ddot \theta_\tn{l} \right)
        + b_\tn{m} \left(\frac{\dot \tau_\tn{des}}{k} + \dot \theta_\tn{l}\right)
        + \tau_\tn{l} \right).
\end{align}

The final two steps of the SEA control are to saturate and low pass filter the
desired joint velocities and feedforward motor torques to ensure safety and
suppress excitatory frequencies. Knee velocities are saturated to lie within
$\unitfrac[\pm 9.95]{rad}{sec}$ and ankle velocities are saturated to lie within
$\unitfrac[\pm 6.96]{rad}{sec}$. Furthermore, if a joint violates specified hard
constraint limits, the desired velocities are saturated to zero to prevent
further constraint violation. For the knee joint, the flexion limit is set to
\unit[90]{degrees} and the extension limit is \unit[-2]{degrees}. For the ankle
joint the flexion/extension limits are set to $\unit[\pm 30]{degrees}$
respectively. The saturated desired motor velocities and feedforward motor
torques are then low pass filtered with second-order Butterworth filters. For
the knee, we use a cutoff frequency of \unit[100]{Hz} and for the ankle, we use
a cutoff frequency of \unit[50]{Hz}.
