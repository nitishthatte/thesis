\section{Introduction}

In \cref{sec:nm_vs_imp} we compared the neuromuscular and impedance controllers.
The results showed that there were reasons for falls with the impedance
controller that did not occur with neuromuscular control. Chief among the types
of falls that occurred with impedance control were knee collapses caused by
premature transitions to the third phase of stance and swing trips, which were
often preceded by a missed transition to the third phase of stance. In contrast,
the neuromuscular control did not suffer from this issue, likely due to its
smooth torque output.

In the neuromuscular controller, the phase of gait is implicitly captured in the
muscle states that emerge from the interplay between multi-segment limb
dynamics, muscle dynamics, and reflexes. A downside to this approach, however,
is that it relies on many parameters that may be difficult to tune, thus
limiting clinical applicability. An alternative approach to achieving smooth
phase estimation during stance is the controller proposed by
\citet{quintero2016preliminary}. This controller explicitly derives a continuous
phase estimate by comparing the hip angle to its integral. However, as we show
in \cref{sec:polar_phase} this approach may be sensitive to step-to-step changes
in gait due as it relies on the integral of the hip angle over a gait cycle .
Recently, \citet{rezazadeh2018phase} eliminated the reliance on the hip integral
by re-introducing discrete state transitions based on thigh angle and velocity
thresholds. However, this approach could face similar robustness issues as the
previously described finite-state impedance control.

Therefore, in this chapter, we propose a new control strategy for lower limb
prostheses that is built on a robust and smooth estimate of the phase of gait
and parameterizes the control outputs in an interpretable manner. We start by
presenting our attempt at implementing the phase-based controller proposed by
\citet{quintero2016preliminary} (\cref{sec:polar_phase}). In \Cref{sec:methods},
we present an Extended Kalman Filter (EKF) that estimates the phase and its rate
of change during the stance portion of gait based on a multitude of sensor
measurements.  We then use sparse Gaussian Process (GP) observation models to
learn relationships between phase and sensor measurements for specific users and
to choose the appropriate control actions for the prosthesis. In
\Cref{sec:results}, we evaluate the performance of the proposed controller with
experiments on able-bodied subjects and a single amputee subject. Finally, in
\cref{sec:discussion} we discuss the results and highlight potential limitations
of this study as well as avenues for future research.
