\section{Introduction}

The number of lower limb amputees in the United States is projected to increase
from 1.6 million in 2005 to 3.6 million in 2050~\citep{ziegler2008estimating}.
Expected causes include increases in the rates of vascular disease, diabetes,
and the size of the elderly population. Prosthetic legs currently prescribed to
these lower limb amputees are mostly passive or semi-passive devices; unlike
human limbs, they cannot produce positive net work over a gait cycle.
Consequently, amputees often suffer from slow walking speeds, high energy
consumption~\citep{waters1976energy}, and an increased risk of
falling~\citep{miller2001prevalence}. Development of active powered prostheses
may help address these gait deficiencies and improve the quality of life for
amputees.

A variety of strategies have been proposed to control active-powered prostheses.
Currently, the most widely used control method for powered transfemoral
prostheses is finite state impedance control. This strategy divides the gait
cycle into several discrete phases, each with a different function mapping from
joint angle and speed to torque~\citep{lawson2014robotic}. This control method
relies on the detection of gait events, such as joint angles crossing
thresholds, to trigger phase transitions that may cause abrupt changes in torque
output as well as unreliable responses to gait disturbances. 

To achieve a more smooth and robust control of lower limb prostheses,
researchers have investigated alternative approaches. One such alternative uses
models of the human neuromuscular system. In this approach, the phase of gait is
implicitly captured in the muscle states that emerge from the interplay between
multi-segment limb dynamics, muscle dynamics, and
reflexes~\citep{eilenberg2010control, thatte2016toward}. A downside to these
approaches, however, is that they often involve many parameters that may be
difficult to tune, thus limiting clinical applicability. Another alternative
approach is exemplified with the phase variable controller proposed by
\citet{quintero2016preliminary}. This controler explicitly derives a continuous
phase estimate by comparing the hip angle to its integral. This approach may be
sensitive to step-to-step changes in gait due to drift in the hip angle integral
term. In later work, \citet{rezazadeh2018phase} eliminated the reliance on the
hip integral by re-introducing discrete state transitions based on thigh angle
and velocity thresholds. However, this approach could face similar robustness
issues as the previously described finite-state impedance control.

Here we propose a control strategy for lower limb prostheses that is built on a
robust and smooth estimate of the phase of gait and does not require a large
number of tuning parameters. In \Cref{sec:methods}, we present an Extended
Kalman Filter (EKF) that estimates the phase and its rate of change during the
stance portion of gait based on a multitude of sensor measurements.  We then use
sparse Gaussian Process (GP) observation models to learn relationships between
phase and sensor measurements for specific users and to choose the appropriate
control actions for the prosthesis. In \Cref{sec:results}, we evaluate the
performance of the proposed controller with experiments on able-bodied subjects
and a single amputee subject. Finally, in \cref{sec:discussion} we discuss the
results and highlight potential limitations of this study as well as avenues for
future research.
