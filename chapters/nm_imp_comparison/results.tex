\section{Results}
\begin{figure}[t]
    \centering 
    \includegraphics[width=\textwidth]{treadmill_vib_user_scores}
    \caption[Average user ratings]{Average user ratings across
    all trials in both the undisturbed and disturbed walking conditions when
    walking without the prosthesis (No Pros) and with the Neuromuscular (NM)
    prosthesis control and impedance (IMP) prosthesis control. Grey bars show
    the mean across subjects.  Statistical significance assessed by paired
    $t$-tests. $*$:~$p < 0.05$, $**$:~$p < 0.01$, $***$:~$p <
    0.001$.}\label{fig:treadmill_user_ratings}
\end{figure}
First, \cref{fig:treadmill_user_ratings} shows the user ratings of the different
conditions. We mandated that users rate the No Prosthesis/No Disturbance case
10/10 so that other conditions could be rated relative to this case. We see that
in both the no disturbance and disturbance cases, neuromuscular control was
rated significantly more preferably than impedance control. Neither control
could match the ratings given to the no prosthesis case. Introduction of the
disturbance caused a significant drop in user rating for all controllers. 

\begin{figure}[b]
    \centering 
    \includegraphics[width=\textwidth]{treadmill_vib_num_falls}
    \caption[Total number of falls]{Total number of falls across all trials in
    both the undisturbed and disturbed walking conditions when walking without
    the prosthesis (No Pros) and with the Neuromuscular (NM) prosthesis control
    and impedance (IMP) prosthesis control. Grey bars show the median number of
    falls across all subjects. Statistical significance assessed by Wilcoxon
    signed-rank test.  $*$:~$p < 0.05$, $**$:~$p <
    0.01$.}\label{fig:treadmill_exp_falls}
\end{figure}
Next, \cref{fig:treadmill_exp_falls} shows the number of falls in each
condition. Here we see that there were significant differences in the median
number of falls between impedance control and no prosthesis walking in the no
disturbance case and both impedance and neuromuscular walking in the disturbance
case. No significant differences were found directly between the neuromuscular
and impedance controllers.

\begin{figure}[t]
    \centering 
    \includegraphics[width=\textwidth]{treadmill_vib_torso_var_x}
    \caption[Torso pitch angle variation]{Torso pitch angle variation. Angle
    variation calculated as the interquartile range of torso angles after the
    median torso angle trajectory over the strides in a trial is subtracted out.
    For the prosthesis trials, we report the average variation across the five
    trials for each condition.  Grey bars show the mean across subjects.
    Statistical significance assessed by paired $t$-tests. $*$:~$p < 0.05$,
    $***$:~$p < 0.001$.}\label{fig:treadmill_exp_torso_var_x}
\end{figure}

\begin{figure}[t]
    \centering 
    \includegraphics[width=\textwidth]{treadmill_vib_torso_var_y}
    \caption[Torso roll angle variation]{Torso roll angle variation. Angle
    variation calculated as the interquartile range of torso angles after the
    median torso angle trajectory over the strides in a trial is subtracted out.
    For the prosthesis trials, we report the average variation across the five
    trials for each condition.  Grey bars show the mean across subjects.
    Statistical significance assessed by paired $t$-tests.$***$:~$p <
    0.001$.}\label{fig:treadmill_exp_torso_var_y}
\end{figure}

\Cref{fig:treadmill_exp_torso_var_x,fig:treadmill_exp_torso_var_y} show the
torso pitch and roll angle variability respectively. We see significant
differences between the no prosthesis and with prosthesis cases as well as the
no disturbance and with disturbance cases. There is also a significant increase
in torso pitch variability with the impedance control compared to the
neuromuscular control in the disturbance case.

\begin{table}[t]
  \begin{center}
    \begin{tabular}{lcc}
      Fall Types & Neuromuscular & Impedance \\
      \midrule
      Fall Forward &  1 &  0 \\
      Fall Backward &  6 &  4 \\
      Fall Left &  1 &  0 \\
      Fall Right &  0 &  3 \\
      Missed Stance / Swing Transition &  3 &  0 \\
      Missed Stance 2 / Stance 3 Transition &  0 &  7 \\
      Knee Collapse & 0 & 15 \\
      Swing Trip & 4 & 12 \\
    \end{tabular}
  \end{center}
  \caption[Tally of observed reasons for falls]{Tally of observed reasons for
  falls across all subjects and across both the undisturbed and disturbed
  walking conditions. Falls were manually classified based on video and logged
  prosthesis data. An individual fall can be assigned to more than one
  reason.}\label{tab:treadmill_exp_fall_reasons}
\end{table}
Finally, \cref{tab:treadmill_exp_fall_reasons} shows a tally of the reasons for
the observed falls with each controller type when using preferred parameters.
We manually determined the reason for each fall by analyzing video recordings,
motion capture data, and logged prosthesis data. The first four categories refer
to general losses of balance resulting in a fall in the four cardinal
directions. Backward falls generally resulted from the treadmill suddenly
stopping when the prosthesis stance leg was still in front of the body, causing
a loss of balance backward. The falls forward, left, and right were generally
more ambiguous in their cause, but may be due to improper leg placement. 

The missed stance/swing transitions in the neuromuscular control were caused
when subjects did not allow the leg angle to cross the $90^\circ$ threshold set
in stance/swing state machine (compare \cref{fig:stance_swing_state_machine}).
The missed stance 2/stance 3 transitions occurred with impedance control if the
user did not dorsiflex the ankle sufficiently to trigger the transition. This
could cause the knee to produce an extension torque in late stance, making it
difficult to enter the swing phase (compare \cref{fig:treadmill_imp_fit}). As
shown in \cref{fig:treadmill_exp_phase_success}, the rate at which the impedance
controller failed to transition through all three stance phases significantly
increased with the introduction of disturbances.
\begin{marginfigure}[0in]
    \centering 
    \includegraphics[width=\textwidth]{phase_success}
    \caption[Fraction of steps for which impedance control successfully
    transitions through all three stance phases]{Fraction of steps for which
    impedance control successfully transitions through all three stance phases.
    Disturbances significantly decrease the transition success rate. Grey bars
    show the mean success rate across all users. Statistical significance
    assessed by paired $t$-test.  $***$:~$p <
    0.001$.}\label{fig:treadmill_exp_phase_success}
\end{marginfigure}

In contrast, the knee collapse fall type was triggered in impedance control if
the user dorsiflexed the ankle too early causing a premature switch to the third
phase of stance. In this phase, knee torque typically trends towards zero to
allow for passive flexion of the knee heading into swing. However, in the case
of a premature switch to the push-off phase, these near-zero knee torques can
cause the knee to suddenly collapse under the user's weight. 

The last cause of falls, trips during swing, occurred when using both
controllers, but 3x more often with impedance control than with neuromuscular
control. Many of the swing trips for impedance control were also preceded by a
missed transition between the second and third phases of stance. Others occurred
when kinematics were drastically changed by the disturbance. For example,
several swing trips occurred after a sudden acceleration of the treadmill caused
the stance step length to dramatically increase, thereby altering kinematics at
toe-off and in swing, and leading to the toe hitting the ground mid-swing.

\begin{figure}[t]
    \centering 
    \includegraphics[width=\textwidth]{treadmill_vib_user_scores_subopt}
    \caption[Comparison of user scores of optimal versus suboptimal parameters
    sets]{Comparison of user scores of optimal versus suboptimal parameters sets
    for the neuromuscular and impedance control strategies. Grey bars show the
    mean user rating across subjects. Statistical significance assessed by
    paired $t$-tests. $**$:~$p <
    0.01$.}\label{fig:treadmill_exp_user_ratings_subopt}
\end{figure}
Finally, we look at the effect of using suboptimal controllers on user ratings
and falls. \Cref{fig:treadmill_exp_num_falls_subopt} shows the median ratings of
each the preferred and suboptimal parameters for each controller. For
neuromuscular control, we see no significant difference between the preferred
controller from day 4 and the suboptimal controllers. In fact for the
neuromuscular control with disturbances, 4 out of 5 users slightly preferred the
suboptimal control from day 5. On the whole, choosing a suboptimal set of
parameters seemed to have a larger effect on impedance control with 4 out of 5
subjects preferring the optimal to suboptimal parameters without disturbances
and all five subjects preferring the optimal impedance parameters to the
suboptimal parameters in the disturbed case.

\Cref{fig:treadmill_exp_num_falls_subopt} shows the median number of falls
garnered by optimal and suboptimal parameters. In the disturbance case, we see
an increase in the median number of falls with the suboptimal parameter sets
over the preferred parameter sets. However, this difference was not significant.

\begin{figure}[b]
    \centering 
    \includegraphics[width=\textwidth]{treadmill_vib_num_falls_subopt}
    \caption[Comparison of number of falls of optimal versus suboptimal
    parameters sets]{Comparison of number of falls of optimal versus suboptimal
    parameters sets for the neuromuscular and impedance control strategies. Grey
    bars show the median number of falls across subjects. Statistical
    significance assessed by Wilcoxon signed-rank
    test.}\label{fig:treadmill_exp_num_falls_subopt}
\end{figure}
