\section{Methods}

\subsection{Parameter Generation}
\begin{margintable}    
    \centering
    \normalsize
    \begin{tabular}{ll}
        \multicolumn{2}{l}{Optimized Parameters} \\
        \midrule
        $F_\tn{max}^\tn{ham}$              & $^{F+}G_\tn{ham}^\tn{ham}$   \\
        $F_\tn{max}^\tn{vas}$              & $^{F+}G_\tn{vas}^\tn{vas}$   \\
        $F_\tn{max}^\tn{gas}$              & $^{F+}G_\tn{gas}^\tn{gas}$   \\
        $F_\tn{max}^\tn{sol}$              & $^{F+}G_\tn{sol}^\tn{sol}$   \\
        $F_\tn{max}^\tn{ta}$               & $^{F-}G_\tn{sol}^\tn{ta}$    \\
        $^\tn{off}l_\tn{ta}^\tn{ta}$       & $^{L+}G_\tn{ta}^\tn{ta}$     \\
        $^\tn{off}\phi_\tn{knee}^\tn{vas}$ & ${^\phi}G_\tn{knee}^\tn{vas}$ \\
        $S_0^\tn{vas}$                     & $\epsilon_\tn{SE}^\tn{ap}$   \\
        $S_0^\tn{ham}$                     & $F_\tn{init}^\tn{ham}$       \\
    \end{tabular}
    \caption[Parameters optimized for parameter set generation for experiment
    comparing neuromuscular and impedance control]{Optimized parameters,
    $\Gamma$. We optimize 18 parameters. $F_\tn{max}^m$ refers to muscle $m$'s
    maximum isometric force, $S_0^m$ is muscle $m$'s pre-stimulation,
    $^\tn{signal} G_n^m$ is the gain on a feedback signal from muscle $n$
    acting on muscle $m$, $\epsilon_\tn{SE}^\tn{ap}$ is the tendon reference
    strain of the ankle plantarflexors (sol and gas) and $F_\tn{init}^\tn{ham}$
    is the initial force in the hamstring MTU at
    heelstrike.}\label{tab:nm_params_treadmill_exp}
\end{margintable}
To obtain suitable parameters for the neuromuscular and impedance control
methods we rely on the dueling bandits optimization approach outlined in
\cref{sec:bandit_optimization}. Whereas in \cref{sec:bandit_optimization} we
optimized control parameters to match gait data at different speeds to achieve a
speed-adaptive control, in this work, we optimize control parameters to match
both undisturbed and disturbed gait in order to obtain robust control
parameters. We use the dataset provided by \citet{moore2015elaborate}, which
provides gait data for undisturbed walking and walking with treadmill velocity
disturbances. 

For the neuromuscular control, we use the black-box CMA-ES optimizer
\citep{hansen2006cma} to obtain parameters that can reproduce the behavior of
each subject in the gait dataset. We optimize the parameters listed in
\cref{tab:nm_params_treadmill_exp} so the model's output torques match those in
the gait dataset. Specifically, we minimize the following cost function:
\begin{align}
    \Gamma &= \argmin_\Gamma \left(\tau_\tn{h} - \tau_\tn{nm} \right)^T
    \left(\tau_\tn{h} - \tau_\tn{nm} \right) + \alpha \xi_\tn{nm}^T \xi_\tn{nm}
\end{align}
where $\tau_\tn{nm}$ and  $\xi_\tn{nm})$ are the torques and muscle activations
respectively generated by the neuromuscular model given the human joint angle
trajectories and model parameters $\Gamma$. $\alpha = 0.01$ is a small constant
we use to help regularize the solutions and prevent muscle stimulations from
saturating.  \Cref{fig:treadmill_nm_fit} shows an example of the fit achieved to
one subject's joint moments.
\begin{figure}[t]
    \centering 
    \includegraphics[width=\textwidth]{nm_fit}
    \caption{Example of fit to subject data achieved by neuromuscular
    model.}\label{fig:treadmill_nm_fit}
\end{figure}

\todo{fix reference}
To generate parameters for impedance control, which is described in full in
\cref{sec:back_impedance}, we follow a two step procedure: In the first step, we
identify appropriate joint angle thresholds that define the impedance
controller's finite state machine transition rules. In the impedance controller,
the transition from phase 1 to phase 2 of stance is based on the knee angle
crossing a threshold. We specify this threshold such that 95\% of steps in a
subject's gait data pass from phase 1 to phase 2. As we use gait data with
disturbances, this procedure automatically sets the threshold such that it
allows for a large degree of gait variation. Next, we identify the ankle angle
threshold that defines the transition between stance phases 2 and 3. Again, we
set this threshold such that 95\% of the steps that made it through the first
transition successfully complete the second transition as well. We set the
thresholds so that 95\% of steps pass through, instead of 100\% of steps, so as
to ignore potential outlier steps.

Once we identify the joint angle thresholds that define state transitions, we
next fit the impedance parameters within each phase. In each phase, the torque
output of the impedance control for a particular joint is 
\begin{align}
    \tau_\tn{imp} &= -k \left( \theta - \theta_0 \right) - b \dot{\theta} \\
        &= \begin{bmatrix} -\theta & -\dot{\theta} & 1 \end{bmatrix}
            \begin{bmatrix} k \\ b \\ k \theta_0 \end{bmatrix} \\
        &= \Theta \vec{k},
\end{align}
where $\Theta$ is a matrix of the subject's joint angles and velocities and
$\vec{k}$ is a vector of the impedance parameters. Therefore, the squared error
between the subject's joint torque and the impedance control model is
\begin{align}
    \epsilon_\tau & = {\left( \tau_\tn{imp} - \tau_\tn{h} \right)}^T 
        \left( \tau_\tn{imp} - \tau_\tn{h} \right) \\
        &= \vec{k}^T \Theta^T \Theta \vec{k} - 2\tau_\tn{h}^T \Theta \vec{k} 
        + \tau_\tn{h}^T \tau_\tn{h}.
\end{align}
To calculate the impedance parameters for each phase we minimize the squared
error subject to the constraints $k>0$ and ${b > 0}$, which ensure that
the resulting impedance models are stable. Finally, to obtain model parameters
that are robust to outlier steps in the dataset, we utilize the RANSAC
procedure, which iteratively solves the above optimization on randomly sampled
subsets of the data in order to classify outliers and fit to inliers only
\citep{fischler1981random}.

\Cref{fig:treadmill_imp_fit} shows an example of the impedance control model
optimized to match one subject's gait data. In this figure, the color of the
lines indicates the phase of gait. We see that the majority of steps fit the
subject's joint moments (grey) well. However, there are a few steps for which
the color of the line and thus the phase does not transition properly.
Consequently, the resulting torque diverges from the human data. This is
expected as the phase transition angles were selected such that 95\% of steps
pass through each phase transition.
\begin{figure}[t]
    \centering 
    \includegraphics[width=\textwidth]{imp_fit}
    \caption[Example of fit to subject data achieved by impedance control
    model]{Example of fit to subject data achieved by impedance control model
    to. Note that green trajectories in knee moment plot and blue trajectories
    in ankle moment plot that do not track subject data are those that did not
    successfully transition to the next phase.}\label{fig:treadmill_imp_fit}
\end{figure}

\subsection{Iterative Learning}\label{sec:treadmill_exp_iterative_learning}
In~\cref{sec:bandit_methods}, in order to compensate for kinematic differences
between the joint angles in the gait dataset and the joint angles of the
prosthesis, we applied hand-tuned offsets to the measured prosthesis joint
angles before calculating the neuromuscular model torques
(\cref{eq:bias_param}). These offsets helped ensure the prosthesis achieved
comfortable levels of ankle dorsiflexion, and prevented knee over extension (or
flexion) during stance. In this experiment, in order to reduce the potential for
bias induced by hand tuning, we take a more systematic, iterative learning
approach to tuning these offsets.

During the iterative learning procedure, a subject walks with each parameter set
for both the neuromuscular and impedance controllers. The knee and ankle angle
trajectories during stance are recorded and after each step, the following
update rules are applied to the knee and ankle joint offsets
\begin{fullwidth}
\begin{align}
    \theta_\tn{knee}^\tn{offset} &\leftarrow \theta_\tn{knee}^\tn{offset} +
    k_\tn{lrn} \left(\theta_\tn{knee}^\tn{ext} - \theta_\tn{knee}^\tn{ext,tgt} \right)
    \left(\theta_\tn{knee}^\textrm{flex} < \theta_\tn{knee}^\tn{flex,max}
    \textrm{~OR~} \theta_\tn{knee}^\tn{ext} > \theta_\tn{knee}^\tn{ext,tgt}\right)\\
    \theta_\tn{ankle}^\tn{offset} &\leftarrow \theta_\tn{ankle}^\tn{offset} +
    k_\tn{lrn} \left(\theta_\tn{ankle}^\tn{flex} - \theta_\tn{ankle}^\tn{flex,tgt} \right),
\end{align}
\end{fullwidth}
where $k_\tn{lrn} = 0.05$ controls the learning rate,
$\theta_\tn{knee}^\tn{ext}$ and $\theta_\tn{knee}^\tn{ext,tgt} = 0^\circ$ are
the measured and target knee extension in mid-stance respectively and
$\theta_\tn{ankle}^\tn{flex}$ and $\theta_\tn{ankle}^\tn{flex,tgt} = 12^\circ$
are the measured and target ankle dorsiflexion in mid-stance respectively. The
conditional terms in the knee iterative learning rule prevent the knee offset
angle from inducing more knee flexion if the knee flexion in early stance,
$\theta_\tn{knee}^\textrm{flex}$, crosses a threshold
$\theta_\tn{knee}^\tn{flex,max} = 10^\circ$.

\subsection{Treadmill Disturbance}\label{sec:treadmill_exp_disturbance}
In our experiment we probe the robustness of the impedance and neuromuscular
prosthesis controllers. To this end, we disturb gait using treadmill velocity
disturbances similar to those in the gait dataset we used to generate parameters
\citep{moore2015elaborate}. During the disturbed walking conditions, the
treadmill velocity is generated as follows: First, random accelerations are
sampled from a zero-mean Gaussian distribution with variance
$\unitfrac[35]{m^2}{s^4}$. These accelerations are saturated to the range,
$\unitfrac[{[-15, 15]}]{m}{s^2}$. Next, the acceleration is integrated to obtain
a velocity signal, and the long-term drift is removed by a $2^\tn{nd}$ order
high-pass filter with a passband edge frequency of \unit[0.5]{hz}. Finally, a
constant offset of \unitfrac[0.8]{m}{s} is applied to the velocity signal, which
is then saturated to the range $\unitfrac[{[0, 3.6]}]{m}{s}$.

\subsection{Experimental Protocol}

In our experiment, we evaluate the robustness, user ratings, and causes for
falls of the neuromuscular and impedance controllers in an experiment with ten
able-bodied subjects wearing the prosthesis via an adaptor. All subjects
provided informed consent to IRB-approved protocols. Subject's participated in
the following six-day procedure:

\todo{also include subject stats}

\begin{days}
    \item\label{list:exp_day_1} \emph{Practice Session} Subjects practiced
    walking on the prosthesis until they could achieve consistent gait without
    the use of hand-rails. Subjects who could not achieve hands free walking by
    the end of the two-hour practice session did not continue with the
    experiment.

    \item\label{list:exp_day_2} \emph{Practice Session} On the second day
    subjects continued to practice walking on the prosthesis without the use of
    hand rails. In addition, on this day subjects practiced walking with the
    disturbance described in \cref{sec:treadmill_exp_disturbance}. This session
    lasted for 2 hours.

    \item\label{list:exp_day_3} \emph{Iterative Learning} Subjects walked with
    each of the nine parameter sets for each controller while the iterative
    learning procedure (\cref{sec:treadmill_exp_iterative_learning}) tuned the
    joint angle offsets. 

    \item\label{list:exp_day_4} \emph{Dueling Bandits Optimization} We performed
    the dueling bandits optimization procedure (\cref{sec:bandit_methods}) to
    find each subject's preferred parameters with both controllers. The order in
    which we optimized the controller was chosen randomly.

    \item\label{list:exp_day_5} \emph{Disturbance Experiment - Practice} We
    performed a practice session for the full disturbance experiment. First,
    subjects walked without the prosthesis at \unitfrac[0.8]{m}{s} for
    \unit[2]{minutes} without disturbances and then \unit[2]{minutes} with the
    treadmill velocity disturbance enabled. After completing these no-prosthesis
    trials, subjects donned the prosthesis and tested the neuromuscular and
    impedance controllers in five rounds of trials that consisted of three
    trials each. In each trial, subjects walked without disturbances for
    \unit[1]{minute} and with disturbances for \unit[1]{minute}. In each round
    of trials, the subjects tested their preferred neuromuscular and impedance
    control parameters along with a set of suboptimal parameters for one
    controller type. Odd numbered subjects tested a suboptimal neuromuscular
    parameter set, while even numbered subjects tested a suboptimal impedance
    parameter set. For the suboptimal parameter set we chose the parameter set
    that ranked $7^\tn{th}$ out of 9 in terms of cumulative Copeland score at
    the end of the dueling bandits tuning procedure.

    \item\label{list:exp_day_6} \emph{Disturbance Experiment - Data Collection}
    The procedure for this day was identical to that of \cref{list:exp_day_5}.
    During these trials, a Vicon motion capture system captured the motion of
    the legs. Additionally, subjects wore an IMU that measured the roll and
    pitch of the torso during walking. During trials, we recorded the number of
    falls (measured as the number of times subjects needed to use the hand rails
    or the ceiling-mounted hardness to recover balance) and the user ratings for
    both the undisturbed and disturbed conditions of each trial. 
\end{days}

We evaluate the robustness of the two control strategies primarily by looking at
the number of falls experienced by each subject in the no disturbance and
disturbance cases for each controller. As a baseline, we also compare to the no
prosthesis case. As a secondary measures of gait robustness, we also measure the
variability of the torso pitch and roll angles. The variability is measured by
subtracting the median torso angle trajectory over the strides in a condition
from the corresponding torso angle trajectories. Then the interquartile range
(IQR) of the median subtracted trajectories is used as the measure of
variability.
