\section{Conclusion} 
\label{sec:conclusion}
This paper presents a model predictive controller that addresses the
problem of trip avoidance for a powered transfemoral prosthesis. Our
controller estimates hip height and hip angle using an Extended
Kalman Filter, predicts future hip angles and hip height using
sparse Gaussian Processes, and plans ankle
and knee trajectories using a fast quadratic program solver. We
tested our system by having an able-bodied subject purposefully lower
their hip on each swing while the prosthesis randomly used either our
controller or a standard minimum jerk controller. We showed that our
system reduced the rate of tripping by 68\% over the standard minimum
jerk controller and reduced the severity of toe-scuffing. To the best
of our knowledge, this controller is the first to incorporate real-time
visual input while planning motion for a prosthesis during gait. 