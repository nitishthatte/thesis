\section{Approach and Contributions}\label{sec:intro_approach_contrib} 

We first investigate whether a novel neuromuscular transfemoral prosthesis
control can improve gait robustness. This control approach seeks to mimic the
underlying dynamics and control of the human neuromuscular system. Unlike the
popular finite state impedance control methods, which we review in
\cref{sec:back_walk_fsm}, during stance, the neuromuscular control approach
provides smooth torque outputs that do not vary drastically due to discrete
phase transitions.  We are motivated to evaluate this control approach by
simulations of biped walking models driven by neuromuscular control that
demonstrate its robustness \citep{song2013integration,song2015neural} and
potential for improvement over finite state impedance control
(\cref{sec:control_sim} \citep{thatte2016toward}) and by previous work
demonstrating the adaptability of neuromuscular control on powered ankle
prostheses \citep{eilenberg2010control,markowitz2011speed}.

To objectively compare the proposed neuromuscular prosthesis control approach to
other methods we required a transfemoral prosthesis capable of accurately
reproducing desired torques. To this end, in \cref{sec:pros_design} we detail
our first contribution: the design of a transfemoral prosthesis that uses series
elastic actuators to achieve accurate torque control and is capable of large
joint torques to allow for behaviors such as running and stumble recovery.  We
evaluate the proposed design by measuring its torque tracking bandwidth, zero
torque tracking capability, and ability to track torques during normal walking.

Also needed to objectively compare control methods, is a method of finding
parameters that suit individual subjects. There are two main issues that must be
overcome to solve this problem. First, it is unclear which metric of gait we
should be optimizing. There are a myriad of qualities necessary for a ``good''
gait including gait robustness, naturalness, energy efficiency, and comfort.
Some of these metrics such as naturalness can be easily quantified, others, such
as robustness and energy efficiency are quantifiable but only with considerable
data, and others, such as comfort, are less easily quantifiable. Even if these
metrics can be quantified their relative importance is unknown and may be unique
to each user. The second issue is that many prosthesis control methods have a
large number of parameters, which makes optimization of their parameters
difficult due to the curse of dimensionality. For example, to find parameters
for the neuromuscular control scheme we propose in 
\cref{sec:nm_control_prosthesis}, we optimize at least 18 parameters.

We present two contributions in \cref{sec:preference_optimization} to tackle
these issues. Both approaches utilize preference feedback between pairs of
control parameters that allow the user to implicitly define the cost function
via their qualitative feedback. In the first approach, we examine a Bayesian
optimization scheme that chooses, at each iteration, a pair of parameter vectors
to compare that is expected to maximize information gained about the location of
the user's optimum. Our experimental results show that this approach is unable
to address problems of large enough dimensionality. Therefore, in the second
contribution, we propose an alternative approach that utilizes offline
optimizations to generate a discrete library of control parameter sets from
which the user chooses their optimum, again through pairwise preference
feedback.

With the prosthesis and parameter selection method in place, in
\cref{sec:nm_vs_imp}, we present the results of an objective comparison between
neuromuscular and impedance control strategies. Based on the simulations results
comparing neuromuscular and impedance control presented \cref{sec:control_sim},
we hypothesized that neuromuscular control would be less susceptible to falls
than impedance control. While we could not verify this hypothesis at the $p <
0.05$ level, we were able to identify failure modes that were unique to
impedance control and found that identifying trips during swing could have a
major impact on gait robustness.

Based on the results of the preceding experiment we make three additional
contributions. The first two, presented in \cref{sec:trip_avoidance} are
motivated by the observation that trips account for a significant number of
falls recorded in the above experiment. Prosthesis swing controls that reduce
the risk of tripping is a largely unexplored area. We explore two distinct
approaches to tackle this problem. In the first, we employ online learning to
train a classifier that can detect the user's obstacle avoidance intent and
switch the prosthesis swing trajectory to a safer one. In the second approach,
we attach a laser distance sensor to the prosthesis and develop an extended
Kalman filter to estimate the user's hip height and orientation. We use this
state estimate to explicitly plan swing trajectories for the knee and ankle
joints that avoid premature toe and heel contact.

Finally, in the last contribution presented in \cref{sec:phase_estimation}, we
propose a new stance control strategy that seeks to rectify the observed issues
with impedance control without the complexity of neuromuscular control.
Moreover, this new proposed strategy seeks to rectify the issues we faced when
implementing a previously proposed phase-based control
\citep{quintero2016preliminary}. In the proposed control method, we obtain a
continuous phase estimate using a multitude of measurements and an extended
Kalman filter. We evaluate the robustness, naturalness, and adaptability of the
proposed control in experiments with seven able-bodied subjects and one amputee
subject.
