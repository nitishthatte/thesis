\subsection{Discussion}\label{sec:completed_comparison_discuss}

Our simulation results suggest that the hybrid neuromuscular control policy can
improve amputee gait stability over existing impedance control methods. An
amputee model walking with a powered prosthesis showed substantial improvements
in balance recovery on rough ground and after swing leg trips when using the
hybrid neuromuscular control policy as opposed to impedance control.  One
possible reason for the improvement is that the proposed controller considers
global leg information such as the target leg angle
(\crefrange{eq:flexphase}{eq:stop}), and it is well known that without placing
the feet into proper target points on the ground, legged systems fail to balance
\citep{townsend1985biped,raibert1986legged,kajita20013d,
seyfarth2002movement,pratt2006capture,wu20133}. A second reason could be that
the design of the swing leg control policy explicitly accounts for large
disturbances to the lower limb dynamics in order to achieve desired leg
placements \citep{desai2012robust}. Neither is the case for current impedance
control policies; however, future research may show that impedance or other
control policies can equally make use of this global information and design
criterion.

Whether the simulation results transfer to amputee gait remains to be
determined. In an initial test with a non-amputee experimenter wearing the
prosthesis via a knee adaptor, we found the hybrid neuromuscular control
reproduces normal walking patterns qualitatively and effectively responds to
disturbances in early and late swing. To understand if these initial results
generalize to amputee locomotion requires further research. First, we only
simulated disturbances in the hardware tests by commanding disturbance torques
to the prosthesis knee. This approach allowed us to apply reproducible
disturbances, but it does not capture real tripping or obstacle encounters,
which will, for instance, exert torques about the hip joint as well.  Second,
the use of the knee adaptor creates abnormal kinematics and inertias and
provides only a loose fit between user and prosthesis. In consequence, we only
tested slow walking at 0.5 m/s holding onto hand rails.

Finally, the simulation and hardware tests captured only a small
portion of the balance disturbances that humans typically encounter
\citep{robinovitch2013video}. Other disturbances may evoke amputee responses
that the simulation model does not capture; especially since it is driven
solely by a reflexive walking controller that ignores conscious interventions.
Already, the hardware experiments revealed that the control's response to
mid-swing disturbances does not match observed human responses and risks
allowing the user to fall. This result suggests the model and corresponding
hardware implementation require additional reflexes or structural changes in
the control to better capture human locomotion and balance recovery. Foot
placement into target points, while beneficial in particular for responding to
early swing disturbances and for rough ground walking, may not be a goal that
the human system prioritizes in response to other disturbances. Identifying
human objectives in these situations could lead to improved leg prosthesis
behaviors independent of the proposed approach, impedance-like approaches, or
other control design approaches.

