\section{Methods}
\subsection{Simulation Environment}\label{sec:complete_simulation_environ}

We studied the performance of the proposed transfemoral prosthesis controller in
a simulation model of a unilateral amputee equipped with the proposed powered
prosthesis. To more accurately model an amputee's anatomy, we severed the femur
of the unimpaired human model \unit[11]{cm} above the knee and attached the
hamstring muscle to the distal end of the shortened bone as recommended in
\citet{brown2012amputation}. This change converts the biarticular hamstring into
a monoarticular muscle that only extends the hip. Next, we attached a model of
the full prosthesis to the severed femur. The prosthesis modeled in this study
is an earlier version of the prosthesis design presented in
\cref{sec:pros_design}, which uses the knee actuator design for both the knee
and ankle joints. Our simulation of the prosthesis models the series elasticity,
electrical dynamics, gear ratios, and resultant reflected inertias of the
actuators and assumes a low-level current-based SEA control achieves desired
torques \citep{pratt1995series}.

As a baseline to compare the performance of the proposed control, we also
simulated the commonly-used impedance control method, which we reviewed in
\cref{sec:back_walk_fsm}, at the behavior level.  Specifically, we implemented
the impedance control presented in \citet{sup2008design} as it tended to perform
better than other versions in our simulations. This control partitions the gait
cycle into four phases. In each phase $i$, the torque of an actuated joint is
governed by an impedance function
\begin{align} 
    \tau_i = -k_{1,i} (\theta - \theta_{1,i}) - k_{2,i} 
        {(\theta - \theta_{2,i})}^3 - b_i \dot{\theta}, 
\end{align} 
where $\theta$ is the joint angle, $\theta_{1,i}$ and $\theta_{2,i}$ are angle
offsets, and $k_{1,i}$, $k_{2,i}$ and $b_i$ are the impedance parameters.  

% Parameter Optimization
\subsection{Controller Optimization for Natural and Robust
Walking}\label{sec:completed_comparison_opt}

For both the hybrid neuromuscular controller and the impedance controller, we
optimized control parameters to search for gaits that appear natural and are
robust to disturbances. For the hybrid neuromuscular model, we optimized 53
parameters that include reflex feedback gains and swing leg control parameters
for both the amputee and prosthesis as well as the SEA control gains. To reduce
the number of parameters to optimize, we used fixed values for many parameters,
such as the muscle properties and prestimulations. For the impedance controller,
we optimized 59 parameters that include the reflex feedback gains and the swing
leg control parameters for the amputee model, and the impedance parameters and
SEA controller gains for the prosthesis. Again to reduce the number of
parameters to optimize, any parameters set to according by \citet{sup2008design}
were fixed to zero during the optimization. 

We relied on the covariance matrix adaptation evolution strategy (CMA-ES)
\citep{hansen2006cma} and performed optimization in two steps. In the first step,
we searched for control parameters that generate a gait with natural kinematics
and kinetics. To this end, we took advantage of the observation that human gait
seems to result from minimizing metabolic energy consumption
\citep{mcneill2002energetics}, and used the cost of transport 
\begin{align}
    \mathit{Cost} = \frac{W}{mgx} + \frac{1}{mgx} \int \left( 
        c_1\tau_\tn{cmd}^2  + c_2 \tau_\tn{limit}^2 \right) \tn{d}t
    \label{eq:costfuncenergy}
\end{align}
as the optimization criterion. In the cost, $W$ accounts for the energy consumption
of both the modeled amputee's muscles and the prosthesis' virtual muscles
according to \citet{umberger2003model}, $\tau_\tn{cmd}$ is the sum of the
torques commanded by the neuromuscular swing control or the impedance control,
$\tau_\tn{limit}$ is the sum of torques produced by the model's mechanical
hard stops, which prevent knee and ankle hyperextension, $m$ is the mass of the
amputee, $g$ is the gravitational acceleration, and $x$ is the distance
traveled in 20 seconds. The hand tuned constants, $c_1 = 0.1$ and $c_2 = 0.01$,
ensure that the terms of the cost function have similar order-of-magnitude. 

We ran the above optimization for 300 iterations and used the best resulting set
of control parameters to seed an optimization for robustness to unexpected
changes in ground height. For this second step, the cost function was
\begin{align}
    \mathit{Cost} = -x + c_3 \int \tau_\tn{limit}^2 \tn{d}t,
\end{align}
rewarding the distance traveled and penalizing joint hyperextension ($c_3 =
0.0005$). Instead of level ground, the simulations evaluating the cost were
performed on terrain that is flat for the first 10 meters (to allow the model to
reach steady walking) and then features steps, spaced one meter apart and drawn
from a uniform random distribution. The width of the distribution grew at a rate
of \unit[2.5]{cm} per meter distance traveled, resulting in steps that grow
progressively rougher the farther the model walks. To avoid overfitting, we
performed the evaluation on five different terrains, resulting in an average
cost. Like in the first step, the optimization was stopped after 300 iterations,
resulting in the final, best set of control parameters.
