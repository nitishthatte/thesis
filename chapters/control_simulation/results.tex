\section{Results}\label{sec:completed_comparison_results}

We evaluated the performance of the proposed control and of impedance control by
having the amputee model walk on terrains that are flat for \unit[10]{meters}
and then feature steps drawn from uniform distributions for another
\unit[90]{meters}. The widths of the distributions are constant but varied among
the terrains to test the control performance on steps of increasing steepness
(\unit[0]{cm} to $\pm \unit[14]{cm}$, \unit[2]{cm} increments, total of 8
terrains). 

\begin{figure}[t]
    \centering
    \includegraphics[width = \columnwidth]{distsWalked}
    \caption[Simulated control performance of prosthesis on rough
    terrain]{Simulated Control performance of prosthesis on rough terrain. The
    distances walked over terrains with different ground roughness are compared
    between the amputee model using a powered knee-ankle prosthesis with
    impedance control (green) and hybrid neuromuscular control (blue) as well as
    with the unimpaired human model (red). Shown are the median and range
    ($25^\textrm{th}$ and $75^\textrm{th}$ percentiles) of the covered distances
    for 50 terrains sampled at each roughness level.}\label{fig:distsWalked}
\end{figure}
\Cref{fig:distsWalked} shows the distances the amputee model walks over 50
trials at each roughness level (proposed neuromuscular control in blue,
impedance control in green). Most of the trials with the impedance-controlled
prosthesis covered the full distance up to a roughness of \unit[2]{cm}. At a
roughness of \unit[4]{cm}, however, the median distance drops to \unit[34]{m},
which further declines as the roughness increases. In contrast, the prosthesis
using the neuromuscular control, allowed the amputee model to walk the full
distance up to a roughness of \unit[6]{cm}. Moreover, neuromuscular control has
a similar distribution of distances walked at a roughness of \unit[8]{cm} as
impedance control has at a roughness of \unit[4]{cm}.

Although controlling the prosthesis with neuromuscular control substantially
improves the robustness of the amputee model on rough terrain, the performance
trails by a large margin that of an unimpaired model (\cref{fig:distsWalked},
red line), for which most of the trials covered the full distance up to a
roughness of \unit[10]{cm}.  Limiting the swing leg placement targets in the
neuromuscular prosthesis control to constant angles may account for some of this
performance gap. In future work, we may overcome this limitation by estimating
the amputee's center of mass velocity and stance ankle position so that the
prosthesis control can take advantage of the full leg placement policy
(\cref{eq:simbicon}). Other sources for the performance gap could stem from
differences in the inertial properties between the prosthesis and the healthy
leg, delay and inaccuracy in the series elastic actuator torque tracking, and
the increased number of parameters in the asymmetric amputee model, which can
reduce the quality of the optimized solutions.

A possible explanation for why the neuromuscular control produces more robust
behavior than impedance control is the former's attempt to mimic the underlying
dynamics and goals of human motor control rather than to track impedance
behavior about a predefined motion for each individual joint. To illustrate this
difference, we subjected the amputee model with both control strategies to a
simulated trip in the form of a $\unit[15]{N \cdot s}$ impulse applied at 5\% of
the undisturbed swing duration. 

\begin{figure*}[t]
    \centering
    \includegraphics{impulseResponseAnnotated}
    \caption[Simulated tripping response of the amputee model with neuromuscular
    and impedance control]{Tripping response of the amputee model with
    neuromuscular (A) and impedance control (B) of the prosthesis. Shown are the
    prosthetic toe trajectories during undisturbed gait (dashed line) and when
    disturbed by a $\unit[15]{N \cdot s}$ impulse (solid line). The
    neuromuscular controller effectively responds to the disturbance and
    maintains a qualitatively similar toe trajectory. The impedance controller
    leads to foot scuffing and an eventual fall.}\label{fig:ImpulseResponses}
\end{figure*}

\Cref{fig:ImpulseResponses}A shows the toe trajectory of the prosthesis using
neuromuscular control both in the undisturbed and disturbed cases. While the
impulse causes a large deviation from the nominal trajectory in early swing, the
controller quickly recovers. From mid-swing onward, the foot follows a
qualitatively similar path, maintains adequate ground clearance, and
successfully reaches a similar foot placement as in the undisturbed case. In
contrast, with impedance control, the prosthesis does not respond adequately
when subjected to the disturbance (\cref{fig:ImpulseResponses}B). This is
illustrated by the prosthesis behavior in mid-swing, during which it does not
react appropriately to maintain ground clearance of the toe. Rather, the
joint-based impedance functions drive the knee into extension prematurely, and
the prosthetic foot scuffs the ground resulting in a trip and subsequent fall.
