\chapter*{Abstract}

Amputees face a number of gait deficits due to a lack of control and power from
their mechanically-passive prostheses. Of crucial importance among these
deficits are those related to balance, as falls and a fear of falling can cause
an avoidance of activity that leads to further debilitation. In this thesis, we
investigate the role that prosthesis control strategies play in maintaining
balance with a powered robotic transfemoral prosthesis. Our approach involves
comparing state-of-the-art prosthesis controllers on a common platform and
learning from this experiment to propose two new prosthesis control strategies
that directly address observed causes of falls in both the swing and stance
phases.

We begin by designing and manufacturing our own powered transfemoral prosthesis
capable of large torques for stumble recovery and accurate reproduction of
desired torques from different control strategies. We also propose a pair of
optimization methods that allow us to select prosthesis control parameters using
qualitative preference feedback from the user.

Next, we test a hypothesis that a stance control approach based on a model of
the human neuromuscular system may help improve gait robustness and user
satisfaction over the commonly used impedance control method. This hypothesis
stems from previous research applying neuromuscular control to simulated biped
models and to powered ankle prostheses that suggests that this approach can
adapt to changes in speed, incline, and rough ground. While our experiment did
not find a significant reduction in falls using neuromuscular control, it did
reveal that a lack of robust gait phase estimation caused a large number of
falls for the impedance control strategy and that both controllers suffered from
trips during swing.

Therefore, we next proposed and tested two new control strategies that directly
address these causes of falls. In the first, we use information from an inertial
measurement unit and a LIDAR distance sensor to estimate the position,
orientation and future trajectory of the hip. This information is then used to
plan trajectories for the prosthesis' knee and ankle that avoid tripping during
swing. Second, we propose using an extended Kalman filter to improve phase
estimation during stance. We show the resulting control strategy significantly
reduced the number of falls compared impedance control when users step on uneven
terrain. These results demonstrate the importance of state estimation for
improving gait stability.
